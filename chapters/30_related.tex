\chapter{Related Work}\label{chap:related}

Present related work that is relevant for the thesis:
For one, work that introduces necessary concepts, and also work that has tried to solve similar problems.

This section is optional.
Instead of presenting related work here, it may be more fitting to present related work in the ``Background'' chapter.
It may also be fitting to split up this chapter into multiple chapters, if there is related work from very different areas, e.g. because the thesis presents a synthesis from two subject areas that are not commonly treated together.

If you choose to include this section, the difference to the ``Background'' chapter would be:
Write here about research that has tried to do similar things, or has tried to solve similar problems, as you do.
The ``Background'' would be for topics that are useful for understanding the thesis and the choices you made, or that explain individual components of your approach or experimental setup. 

\section{Example Text}

There are several branches of work that have previously investigated ways in which cats can improve and optimize their sleeping place selection.

Firstly, there is a plethora of literature on feline behavior.
\citet{whiskers1997art} laid the groundwork with their monumental publication, \citetitle{whiskers1997art}.
This seminal piece, written by cats, for cats, discussed the dynamics of curling into the most comfortable positions and the importance of surface texture.
We build upon \citeauthor{whiskers1997art}'s principles by incorporating machine learning to predict optimal textures and positions based on a given environment.

In the context of comfort prediction models, \citet{fluffytail2002cozyfinder} proposed an early algorithm, dubbed \emph{CozyFinder2000}.
The algorithm exploited the physics of fabric thread counts and surface softness, using rudimentary decision trees to make predictions.
Our work extends CozyFinder2000 by employing sophisticated deep learning techniques, resulting in an advanced algorithm that we have named “The Ultimate Cat Nap Optimizer”.

Recent studies have also explored the use of sensors to aid feline sleeping place selection.
\citet{paws2019tracking} used infrared sensors attached to cat collars to track preferred resting spots.
Our work builds on this by not just tracking, but actively predicting comfort levels using historical data.

Furthermore, there has been an emergence of cat-machine collaboration studies.
\citet{mittens2021catbots} investigated the use of \emph{Cat-bots} to pre-warm resting places.
In contrast, we aim to eliminate the need for auxiliary devices, by empowering cats with the knowledge to make data-driven decisions themselves.

On the machine learning front, \citet{purrsington2023csn} recently introduced Convolutional Sleeping Networks (CSN) which use LSTMs \citep{hochreiter97} to analyze different sleeping surfaces.
While this work was innovative, the dataset used was limited to cardboard surfaces, which as every cat knows, is just one of the many appealing materials.
Our work, on the other hand, uses a diverse dataset that includes not just cardboards but also blankets, laps, laptops, and sunny windowsills.

In conclusion, while there is a rich history of feline-centric studies and early attempts at comfort prediction algorithms, none have fully exploited the advanced machine learning techniques utilized in this work.
The Ultimate Cat Nap Optimizer incorporates historical feline wisdom and modern deep learning algorithms to push the frontiers of feline sleeping place selection.
