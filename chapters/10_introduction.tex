\chapter{Introduction}\label{chap:introduction}

If you are using this template, make sure to also read the \texttt{README.md}!

The introduction should motivate the thesis and give some background info that is necessary to understand the problems involved.
Dividing it into sections is optional, particularly if the sections would end up being very short.
The proposed section headings here should give some orientation about what to include.

After giving some background and motivation, the introduction should make it clear what the central objective being investigated in the thesis is.
This could be how well a newly developed algorithm works compared to other algorithms, or how existing methods compare against each other when compared with respect to a novel aspect.

\section{Motivation}

Write about the overall motivation for the thesis.
This should start in general terms, with a sentence or two introducing the overall subject area, and then be going more in depth about subfields relevant to the problem.

\section{Problem}

Describe, what kind of problem is the method in the thesis is about to solve, and why it is relevant.

\section{Approach}

In many cases, the thesis will be about a new approach to a known problem.
In this case, describe the approach in reasonably general terms, without getting too technical.
The reader should understand what the new idea is that is being used.

If the thesis is more about highlighting a new aspect of known methods, e.g.\ evaluating runtime complexity of methods for which runtime was not systematically measured in the literature, this should be described here instead (possibly with a title that is not ``Approach'').

It may help to define various ``research questions'' that are being investigated.
These can even be numbered and referred to in later parts of the text, e.g.\ to describe which experiments were conducted to answer which kinds of questions.
This is more relevant if the thesis is about benchmarking and comparing different existing algorithms.
If the thesis instead presents a new algorithm, then writing out a research question to the terms of ``is my algorithm better than others?'' is rather silly.

If the thesis developed novel algorithms, this is also a good place to refer to them by name.
E.g.\ ``We present UPDOG (`Unsupervised Probabilistic Dynamic Optimization with Gradients'), which...''

\section{Outline}

Outline of the thesis.
What is in each chapter? What is the structure of the thesis?
E.g.\ ``\cref{chap:background} presents the theoretical background that is relevant for the thesis, particularly mathematical concepts and algorithms''.
List chapters in order, and only list \emph{future} chapters, not the introductory chapter.

The outline is \emph{optional}, and whether to include it is a matter of taste.
Some people think it is redundant to have this.


