\chapter{Conclusion and Outlook}\label{chap:conclusion}

\section{Conclusion}

Statement of the main results and their implications.
What have we learned from all of this?
Should relate to the research questions(s) or hypothesis(es) stated in the introduction.
This part should not contain any new experimental results that are not already mentioned in the thesis!

\section{Outlook}

What could be done in the future?
What are the next steps?
What are the open questions?
What are the limitations of the current work?
This part can be merged with the conclusion.

Note that ``limitations'' refers to (theoretical) limitations of what can be concluded:
What kinds of conclusions can \emph{not} be drawn, e.g.\ because the datasets that were used differ from real-world data, or because various variables were held constant and their effect were not analysed?
This does \emph{not} mean limitations that you were placed under, e.g.\ limited availability of cluster CPU time---choices you had to make because of ``limited'' resources are described in the ``Experiments'' chapter.
You \emph{can}, however, write about how conclusions about e.g.\ large datasets can not be drawn from your results since they only consider small datasets, a choice you had to make because of limited CPU resources.